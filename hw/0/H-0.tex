\documentclass{article}
\usepackage{../fasy-hw}
\usepackage{ wasysym }

%% UPDATE these variables:
\renewcommand{\hwnum}{0}
\title{Discrete Structures, Homework 0}
\author{\todo{Your Name Here}}
\collab{n/a}
\date{due: 15 January 2021}

\begin{document}

\maketitle

This homework assignment should be
submitted as a single PDF file both to D2L and to Gradescope.

General homework expectations:
\begin{itemize}
    \item Homework should be typeset using LaTex.
    \item Answers should be in complete sentences and proofread.
    \item You will not plagiarize.
    \item List collaborators at the start of each question using the
        \texttt{collab} command.
\end{itemize}

% ============================================
% ============================================
\nextprob{Getting to Know You}
\collab{n/a}
% ============================================
% ============================================

Answer the following questions:
\begin{enumerate}
    \item What is your elevator pitch?  Describe yourself in 1-2
        sentences.
        \paragraph{Answer} \todo{answer here}

    \item What was your favorite college class so far, and why?
        \paragraph{Answer} \todo{answer here}

    \item What was your least favorite college class so far, and why?
        \paragraph{Answer} \todo{answer here}

    \item Why are you interested in taking this course? (If your answer is
        `because I am required to by my major/minor', perhaps answer the
        alternative question: Why are you in your major?)
        \paragraph{Answer} \todo{answer here}

    \item What is your biggest academic or research goal for this semester (can
        be related to this course or not)?
        \paragraph{Answer} \todo{answer here}

    \item What do you want to do after you graduate?
        \paragraph{Answer} \todo{answer here}

    \item What was the most challenging aspect of blended or online courses?
        \paragraph{Answer} \todo{answer here}

    \item What do you like about blended or online courses?
        \paragraph{Answer} \todo{answer here}

\end{enumerate}

% ============================================
% ============================================
\nextprob{Administrative Tasks}
\collab{n/a}
% ============================================
% ============================================

Please do the following:
\begin{enumerate}
    \item Write this homework in LaTex. This will not be strictly enforced for
        this homework, but it is strongly encouraged.  Future homeworks will not
        be graded if they are not typeset in LaTex.
    \item Update your photo on D2L to be a recognizable headshot of you.
    \item Sign up for the class discussion board.
\end{enumerate}

\paragraph{Answer}


\todo{write a statement here that confirms that you have completed these tasks}

% ============================================
% ============================================
\nextprob{Plagiarism}
\collab{n/a}
% ============================================
% ============================================


In this class, please properly cite all resources that you use.  To refresh your
memory on what plagiarism is, please complete the plagiarism tutorial found
here: \url{http://www.lib.usm.edu/plagiarism_tutorial}.  If you have observed
plagiarism or cheating in a classroom (either as an instructor or as a student),
explain the situation and how it made you feel.  If you have not experienced
plagiarism or cheating or if you would prefer not to reflect on a personal
experience, find a news article about plagiarism or cheating and explain how you
would feel if you were one of the people involved.

\paragraph{Answer}

\todo{your answer goes between these lines}

% ============================================
% ============================================
\nextprob{Exams}
\collab{n/a}
% ============================================
% ============================================

I am exploring various options for exams for this semester: take-home,
in-person, synchronous online.  If you have any comments about what worked or
did not work in previous semesters with respect to classes in blended and online
settings, please share that here.

\paragraph{Answer}


\todo{your answer goes between these lines}




% ============================================
% ============================================
\nextprob{Terminology}
\collab{\todo{}}
% ============================================
% ============================================

Sometimes concepts are taught more than once throughout the curriculum.  Each
time you encounter a concept, your understanding of it is deepened.
For each of the terms or statements below, describe in your own words what they
mean.  This will not be graded for correctness, just whether you have done it or
not.  Answering these to the best of your ability will help the instructor and
TA understand the base knowledge of the students in this class.
I encourage you to meet with a partner or two to refresh yourself on what these
terms mean (if you do, be sure to update the \texttt{collab} command
above!).  However, please keep the web searches to a minimum for this one!  It
is acceptable to answer `I have not heard of this term' or `I have heard of
this, but do not remember what it means.'
\begin{enumerate}
    \item $f(n)$ is $O(n^2)$.
        \paragraph{Answer}
        \todo{put your definition here}
    \item $f(n)$ is $O(g(n))$.
        \paragraph{Answer}
        \todo{put your definition here}
    \item $f(n)$ is $\Omega(n^3)$.
        \paragraph{Answer}
        \todo{put your definition here}
    \item $f(n)$ is $\Theta(n\log n)$.
        \paragraph{Answer}
        \todo{put your definition here}
    \item Binomial Coefficients
        \paragraph{Answer}
        \todo{put your definition here}
    \item Four Color Theorem
        \paragraph{Answer}
        \todo{put your definition here}
    \item Graph
        \paragraph{Answer}
        \todo{put your definition here}
    \item Modus Ponens
        \paragraph{Answer}
        \todo{put your definition here}
    \item Proof by Counter-example
        \paragraph{Answer}
        \todo{put your definition here}
    \item Proof by Example
        \paragraph{Answer}
        \todo{put your definition here}
    \item Proof by Induction
        \paragraph{Answer}
        \todo{put your definition here}
    \item Recurrence Relation
        \paragraph{Answer}
        \todo{put your definition here}
    \item Recursive Algorithm
        \paragraph{Answer}
        \todo{put your definition here}
    \item Searching Algorithms
        \paragraph{Answer}
        \todo{put your definition here}
    \item Sorting Algorithms
        \paragraph{Answer}
        \todo{put your definition here}
    \item Tree
        \paragraph{Answer}
        \todo{put your definition here}
\end{enumerate}

% ============================================
% ============================================
\nextprob{Real Numbers}
\collab{\todo{}}
% ============================================
% ============================================

Review the Properties of Real Numbers in Appendix A.  If any are unfamiliar or
confusing, please post a question in the group discussion board.  In the
write-up, write the following: `I have reviewed all properties of real numbers
in Appendix A.`

\paragraph{Answer}

\todo{confirm that you have read this section}


% ============================================
% ============================================
\nextprob{Georg Cantor}
\collab{\todo{}}
% ============================================
% ============================================

Write a short (1-2 paragraph) biography of Georg Cantor.
\textbf{In your own words}, describe who they are and why they are important in
the history of computer science.  If you use external resources, please provide
proper citations.

\paragraph{Answer}

% ============================================

\todo{your answer goes between these lines}

% ============================================

\end{document}

