\documentclass{article}
\usepackage{../fasy-hw}

%% UPDATE these variables:
\renewcommand{\hwnum}{3}
\title{Discrete Structures, Homework \hwnum}
\author{\todo{Your Name Here} (\todo{your discord handle here})}
\collab{n/a}
\date{due: 19 February 2021}

\begin{document}

\maketitle

This homework assignment should be
submitted as a single PDF file both to D2L and to Gradescope.

General homework expectations:
\begin{itemize}
    \item Homework should be typeset using LaTex.
    \item Answers should be in complete sentences and proofread.
    \item You will not plagiarize.  \item List collaborators at the start of each question using the \texttt{collab} command.
    \item Put your answers where the \texttt{todo} command currently is (and
        remove the \texttt{todo}, but not the word \texttt{Answer}).
\end{itemize}

% ============================================
% ============================================
\collab{n/a} \nextprob{Negations}
% ============================================
% ============================================
Negate the following statements:

\begin{enumerate}

    \item Each ``Clean 'Cat Kit''  contains a cloth mask and a refillable hand
        sanitizer.

        \paragraph{Answer}
        \todo{your answer here}

    \item There exists a boat docked in New Jersey that I have steered.

        \paragraph{Answer}
        \todo{your answer here}

    \item There exists an island in the Ohio River with a bowling alley and a
        university track field.

        \paragraph{Answer}
        \todo{your answer here}

    \item Both my sister and I can climb every route at Spire.

        \paragraph{Answer}
        \todo{your answer here}

\end{enumerate}

% ============================================
% ============================================
\collab{n/a} \nextprob{Definitions}
% ============================================
% ============================================
Use the definitions provided in the course textbook to prove that every prime
number except~$2$ is odd.

\paragraph{Answer}

\todo{your answer here}

%
% ============================================
% ============================================
\collab{\todo{}}
\nextprob{Four Colors Suffice}
% ============================================
% ============================================
Read Chapters $2$ and $3$ of \emph{Four Colors Suffice} and answer the following questions:

\begin{enumerate}

    \item Who are the Austrian(s) mentioned in Chapters $1$--$3$, and what was their
        contribution mentioned in the book?

        \paragraph{Answer}
        \todo{your answer here.}

    \item Write a statement of the four color theorem using a universal
        quantifier.

        \paragraph{Answer}
        For each map $m$, \todo{ complete this sentence ...}

    \item What is the definition of a $k$-coloring of a graph?

        \paragraph{Answer}
        \todo{your answer here.}

    \item Prove or disprove: all planar graphs are three-colorable.

        \paragraph{Answer}
        \todo{}

    \item Assuming the four color theorem holds, prove or disprove: six colors
        suffice to color a plane graph.

        \paragraph{Answer}
        \todo{}

    \item Give an example of a map with at least five faces that has a
        two-coloring.  Be sure to provide a coloring as evidence that the map is
        two-colorable.

        \paragraph{Answer}
        \todo{your answer here.}

    \item Euler's formula states that if we have a map on the sphere or plane
        and count the exterior face as a face, then F-E+V=2.  Does this equation
        hold if the map is drawn on a M\"obius band? Why or why not? (Note:
        here, the boundary of the M\"obius band must be represented in the graph
        defining the map, and no ``country'' can be on the same side of a single
        edge.)

        \paragraph{Answer}
        \todo{your answer here.}

    \item In your own words, explain the joke: ``A topologist cannot tell the
        difference between a coffee cup and a donut.''  You are encouraged to
        use Wikipedia to formulate your answer, but be sure to cite sources.

        \paragraph{Answer}
        \todo{your answer here.}

\end{enumerate}

% ============================================
% ============================================
\collab{\todo{}}
\nextprob{Ada Lovelace}
% ============================================
% ============================================

Write a short (1-2 paragraph) biography of Ada Lovelace.
\textbf{In your own words}, describe who they are and why they are important in
the history of computer science.  If you use external resources, please provide
proper citations (see the `hw/bib-ex` folder for examples of how to use
citations). If you do not use external sources, please write ``I did not
use any sources to write this biography'' as the last sentence of the
biography.

\paragraph{Answer}

\todo{your answer goes between these lines}

%% ... the bibliography
\newpage
\bibliographystyle{acm}
\bibliography{biblio}

\end{document}

