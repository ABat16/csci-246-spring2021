\documentclass{article}
\usepackage{../fasy-hw}

%% UPDATE these variables:
\renewcommand{\hwnum}{4}
\title{Discrete Structures, Homework \hwnum}
\author{\todo{Your Name Here} (\todo{your discord handle here})}
\collab{n/a}
\date{due: 5 March 2021}

\begin{document}

\maketitle

This homework assignment should be
submitted as a single PDF file both to D2L and to Gradescope.

General homework expectations:
\begin{itemize}
    \item Homework should be typeset using LaTex.
    \item Answers should be in complete sentences and proofread.
    \item You will not plagiarize.
    \item List collaborators at the start of each question using the \texttt{collab} command.
    \item Put your answers where the \texttt{todo} command currently is (and
        remove the \texttt{todo}, but not the word \texttt{Answer}).
\end{itemize}


% ============================================
% ============================================
\collab{\todo{}} \nextprob{Good Proofs}
% ============================================
% ============================================

Look through proofs in this textbook, or other books / papers.  Define five
qualities that you think are common among good proofs. Provide citations to
examples.


\paragraph{Answer}

\todo{your answer here}



% ============================================
% ============================================
\collab{\todo{}} \nextprob{Max of a Subset}
% ============================================
% ============================================

Let $(B,\leq)$ be a totally ordered finite set. Prove the following
statement: For all subsets $A \subseteq B$, the following inequality
holds: $\max(A) \leq \max(B)$.

\paragraph{Answer}

\todo{your answer here}

% ============================================
% ============================================
\collab{\todo{}} \nextprob{Fibonacci}
% ============================================
% ============================================

The Fibonacci numbers are defined as follows:
$$
    F_i = \begin{cases}
            1 & i \in \{1,2\} \\
            F_{n-1}+F_{n-2} & \text{otherwise}
          \end{cases}
$$

Prove $\sum_{i=1}^n F_i = F_{n+2}-1$.

\paragraph{Answer}

\todo{your answer here}

% ============================================
% ============================================
\collab{\todo{}} \nextprob{US Coins}
% ============================================
% ============================================

Consider the four smallest denominations of US coins: $D=\{1,5,10,25\}$.  Prove, using
induction, that, for each $n \geq 1$, you can make $n$ cents using at most four
pennies.

\paragraph{Answer}

\todo{your answer here}

% ============================================
% ============================================
\collab{\todo{}} \nextprob{Four Colors Suffice}
% ============================================
% ============================================

Read Chapters $4$ and $5$ of \emph{Four Colors Suffice}.

Use a ``minimal criminal'' argument to prove that if an edge is removed from a
tree, then the resulting graph has two connected components.

        \paragraph{Answer}

        \todo{your answer here}

% ============================================
% ============================================
\collab{\todo{}}
\nextprob{Leonhard Euler}
% ============================================
% ============================================

Write a short (1-2 paragraph) biography of Leonhard Euler.
\textbf{In your own words}, describe who they are and why they are important in
the history of computer science.

If you use external resources, please provide
proper citations. If you do not use external sources, please write ``I did not
use any sources to write this biography'' as the last sentence of the
biography.

\paragraph{Answer}

\todo{your answer here}

% %% ... the bibliography
% \newpage
% \bibliographystyle{acm}
% \bibliography{biblio}

\end{document}

