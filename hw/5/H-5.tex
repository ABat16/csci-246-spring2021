\documentclass{article}
\usepackage{../fasy-hw}

%% UPDATE these variables:
\renewcommand{\hwnum}{5}
\title{Discrete Structures, Homework \hwnum}
\author{\todo{Your Name Here} (\todo{your discord handle here})}
\collab{n/a}
\date{due: 19 March 2021}

\begin{document}

\maketitle

This homework assignment should be
submitted as a single PDF file both to D2L and to Gradescope.

General homework expectations:
\begin{itemize}
    \item Homework should be typeset using LaTex.
    \item Answers should be in complete sentences and proofread.
    \item You will not plagiarize.
    \item List collaborators at the start of each question using the \texttt{collab} command.
    \item Put your answers where the \texttt{todo} command currently is (and
        remove the \texttt{todo}, but not the word \texttt{Answer}).
\end{itemize}


% ============================================
% ============================================
\collab{\todo{}} \nextprob{Colors}
% ============================================
% ============================================

One thing that we need to consider as computer scientists is making our products
(software, technical papers) accessible to a wide range of people. When
designing GUIs or writing technical papers (e.g., journal papers or even
homework solutions), explain five things that you could do to make your
technical write-ups, website, or GUI products more accessible to people who
might be colorblind or colorweak (or have a bad computer screen).


\paragraph{Answer}

\todo{your answer here}



% ============================================
% ============================================
\collab{\todo{}} \nextprob{The Complete Bipartite Graph $K_{n,n}$}
% ============================================
% ============================================

How many edges does the complete bipartite graph $K_{n,n}$ have?  Make your
conjecture, then prove that it is correct.

Bonus: Instead, prove the more general case:
what is the number of edges in $K_{n,m}$?

\paragraph{Answer}

\todo{your answer here}




% ============================================
% ============================================
\collab{\todo{}} \nextprob{Four Colors Suffice}
% ============================================
% ============================================

Read Chapters $7$ and $8$ of \emph{Four Colors Suffice}.

\begin{enumerate}

    \item In the Four Colors Suffice book, we saw the definition of Euler's
        Formula for a finite decomposition of a Sphere or 2-plane into vertices,
        edges, and faces.  What is the other formula known as Euler's formula?

        \paragraph{Answer}

        \todo{your answer here}



    \item  Consider the following construction: Start with a solid cube.  Then, slice
        off a small region around eachvertex (image you have a sharp knife, so you take
        off a tetrahedron at each corner).  How many vertices, edges, and faces are on
        the surface of this object before and after this operation? What polyhedron is this?

        \paragraph{Answer}

        \todo{your answer here}




    \item Draw a projection of the octahedron onto the plane such that edges only
        intersect at vertics.  Can every polyhedron be drawn in such a way?

        \paragraph{Answer}

        \todo{your answer here}


\end{enumerate}

% ============================================
% ============================================
\collab{\todo{}}
\nextprob{Fran Allen}
% ============================================
% ============================================

Write a short (1-2 paragraph) biography of Fran Allen.
\textbf{In your own words}, describe who they are and why they are important in
the history of computer science.

If you use external resources, please provide
proper citations. If you do not use external sources, please write ``I did not
use any sources to write this biography'' as the last sentence of the
biography.

\paragraph{Answer}

\todo{your answer here}

% %% ... the bibliography
% \newpage
% \bibliographystyle{acm}
% \bibliography{biblio}

\end{document}

