\documentclass{article}
\usepackage{../fasy-hw}

%% UPDATE these variables:
\renewcommand{\hwnum}{5b}
\title{Discrete Structures, Homework \hwnum}
\author{\todo{Your Name Here} (\todo{your discord handle here})}
\collab{n/a}
\date{due: 29 March 2021 (Monday)}

\begin{document}

\maketitle

This homework assignment should be
submitted as a single PDF file both to D2L and to Gradescope.

General homework expectations:
\begin{itemize}
    \item Homework should be typeset using LaTex.
    \item Answers should be in complete sentences and proofread.
    \item You will not plagiarize.
    \item List collaborators at the start of each question using the \texttt{collab} command.
    \item Put your answers where the \texttt{todo} command currently is (and
        remove the \texttt{todo}, but not the word \texttt{Answer}).
\end{itemize}

{\color{blue} This homework is a resubmission of previously assigned homework
questions.  We encourage you to use the previous feedback to improve your
solutions to these problems, as well as to discuss the solutions with each other.
As usual, don't hesitate to ask questions! And, the write-up that you submit
MUST be in your own words.  In
this homework, there are eleven proofs (some are grouped together as one
question).  Each question will be graded using the following scheme:
\begin{itemize}
    \item No credit (+0pts). Either no solution or a major logical error was found
        (e.g., started with what needed to be proven).
    \item Low Pass (+6pts). Some of proof present, but either skips a step (or
        two) or
        has multiple small errors.
    \item Pass (+8pts). Mostly correct. May have small errors.
    \item High Pass (+10pts). Proof is exemplary.
\end{itemize}
}


% ============================================
% ============================================
\nextprob{(1-3) A Proof}
\collab{\todo{}}
% ============================================
% ============================================

Prove that $6\Z \subset 2\Z$.

\paragraph{Answer}

\todo{your answer here}

% ============================================
% ============================================
\collab{\todo{}}
\nextprob{(2-2) Existential Statements}
% ============================================
% ============================================

Are the following statements true or not true?    Prove or disprove.

\begin{enumerate}

    \item All even integers are equal to an odd integer plus one.

        \paragraph{Answer}
        \todo{be sure to write this in if/then format first.  Then, prove or
        disprove.}

    \item All horses are the same color.

        \paragraph{Answer}
        \todo{be sure to write this in if/then format first.  Then, prove or
        disprove.}

\end{enumerate}

% ============================================
% ============================================
\collab{n/a} \nextprob{Definitions}
% ============================================
% ============================================
Use the definitions provided in the course textbook to prove that every prime
number except~$2$ is odd.

\paragraph{Answer}

\todo{your answer here}

% ============================================
% ============================================
\collab{\todo{}}
\nextprob{(3-3) Four Colors Suffice}
% ============================================
% ============================================
Read Chapters $2$ and $3$ of \emph{Four Colors Suffice} and answer the following questions:

\begin{enumerate}
    \item[4.] Prove or disprove: all plane graphs (maps) are three-colorable.

        \paragraph{Answer}
        \todo{}

    \item[5.] Assuming the four color theorem holds, prove or disprove: six colors
        suffice to color a plane graph.

        \paragraph{Answer}
        \todo{}

    \item[7.] Euler's formula states that if we have a map on the sphere or plane
        and count the exterior face as a face, then F-E+V=2.  Does this equation
        hold if the map is drawn on a M\"obius band? Why or why not? (Note:
        here, the boundary of the M\"obius band must be represented in the graph
        defining the map, and no ``country'' can be on the same side of a single
        edge.)

        \paragraph{Answer}
        \todo{your answer here.}

\end{enumerate}




% ============================================
% ============================================
\collab{\todo{}} \nextprob{(4-2) Max of a Subset}
% ============================================
% ============================================

Let $(B,\leq)$ be a totally ordered finite set. Prove the following
statement: For all nonempty subsets $A \subseteq B$, the following inequality
holds: $\max(A) \leq \max(B)$.

\paragraph{Answer}

\todo{your answer here}

% ============================================
% ============================================
\collab{\todo{}} \nextprob{(4-3) Fibonacci}
% ============================================
% ============================================

The Fibonacci numbers are defined as follows:
$$
    F_i = \begin{cases}
            1 & i \in \{1,2\} \\
            F_{i-1}+F_{i-2} & \text{otherwise}
          \end{cases}
$$

Prove $\sum_{i=1}^n F_i = F_{n+2}-1$.

\paragraph{Answer}

\todo{your answer here}

% ============================================
% ============================================
\collab{\todo{}} \nextprob{(4-4) US Coins}
% ============================================
% ============================================

Consider the four smallest denominations of US coins: $D=\{1,5,10,25\}$.  Prove, using
induction, that, for each $n \geq 1$, you can make $n$ cents using at most four
pennies.

\paragraph{Answer}

\todo{your answer here}

% ============================================
% ============================================
\collab{\todo{}} \nextprob{(4-5) Four Colors Suffice}
% ============================================
% ============================================

Read Chapters $4$ and $5$ of \emph{Four Colors Suffice}.

Use a proof by contradiction to prove that if an edge is removed from a
tree, then the resulting graph has two connected components.

EC:
Use a ``minimal criminal'' argument to prove this.

        \paragraph{Answer}

        \todo{your answer here}

% %% ... the bibliography
% \newpage
% \bibliographystyle{acm}
% \bibliography{biblio}

\end{document}

