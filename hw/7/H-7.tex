\documentclass{article}
\usepackage{../fasy-hw}

%% UPDATE these variables:
\renewcommand{\hwnum}{7}
\title{Discrete Structures, Homework \hwnum}
\author{\todo{Your Name Here} (\todo{your discord handle here})}
\collab{n/a}
\date{due: 16 April 2021}

\begin{document}

\maketitle

This homework assignment should be
submitted as a single PDF file both to D2L and to Gradescope.

General homework expectations:
\begin{itemize}
    \item Homework should be typeset using LaTex.
    \item Answers should be in complete sentences and proofread.
    \item You will not plagiarize, nor will you share your written solutions
        with classmates.
    \item List collaborators at the start of each question using the \texttt{collab} command.
    \item Put your answers where the \texttt{todo} command currently is (and
        remove the \texttt{todo}, but not the word \texttt{Answer}).
\end{itemize}


% ============================================
% ============================================
\collab{\todo{}} \nextprob{Graphs}
% ============================================
% ============================================

Often, in order to transform real-world problems into ones that can be analyzed
on computers, you need to design a representation of the data that helps
illuminate the patterns.  One common representation is a graph.  Suppose you own
a movie store.  You have records of every movie purchased, how much it was
purchsed for, and who purchased it.  Describe two different graphs that you can
create to represent this data.  Please make sure that the nodes in the two
graphs represent different things.

\paragraph{Answer}

\todo{your answer here}


% ============================================
% ============================================
\collab{\todo{}} \nextprob{Equivalence Class}
% ============================================
% ============================================

Define a relation $R$ between all simple graphs where two graphs $g$ and $h$ are
related (denoted $gRh$) if and only if $g$ and $h$ have the same number of
connected components.

\begin{enumerate}

    \item Prove that this is an equivalence relation.

        \paragraph{Answer}

        \todo{your answer here}

    \item Describe a scenario where you might use this equivalence relation.

        \paragraph{Answer}

        \todo{your answer here}

\end{enumerate}

% ============================================
% ============================================
\collab{\todo{}} \nextprob{Pseudocode}
% ============================================
% ============================================

Recall the binary search algorithm.

\begin{enumerate}
    \item Using the algorithm/algorithmic environment,
        give pseudocode using a for loop.

        \paragraph{Answer} My algorithm for binary search using a for loop is given in \algref{forloop}.

        \begin{algorithm}
            \caption{\textsc{BinarySearchFor}$(A)$}\label{alg:forloop}
            \begin{algorithmic}
                \State \todo{pseudocode here}
            \end{algorithmic}
        \end{algorithm}

    \item Using the algorithm/algorithmic environment, give pseudocode using a while loop.

        \paragraph{Answer}

        \todo{your answer here}

    \item Using the algorithm/algorithmic environment, give pseudocode using
        recursionn.

        \paragraph{Answer}

        \todo{your answer here}

    \item What is the loop invariant of your second algorithm? (Proofs are not
        necessary, just stating the LI is required here.  As usual, for partial
        credit of an incorrect answer, reasoning will need to be justified).

        \paragraph{Answer}

        \todo{your answer here}
\end{enumerate}

% ============================================
% ============================================
\collab{\todo{}} \nextprob{Four Colors Suffice}
% ============================================
% ============================================

Read Chapter $11$ of \emph{Four Colors Suffice}.

\begin{enumerate}

    \item What is a proof?

        \paragraph{Answer}

        \todo{your answer here}


    \item Choose one concept that was described in both FCS and in Epp.
        Compare and contrast their explanations of the concept.

        \paragraph{Answer}

        \todo{your answer here}

\end{enumerate}

% ============================================
% ============================================
\collab{\todo{}}
\nextprob{Thomas Bayes}
% ============================================
% ============================================

Write a short (1-2 paragraph) biography of Thomas Bayes.
\textbf{In your own words}, describe who they are and why they are important in
the history of computer science.

If you use external resources, please provide
proper citations. If you do not use external sources, please write ``I did not
use any sources to write this biography'' as the last sentence of the
biography.

\paragraph{Answer}

\todo{your answer here}

% %% ... the bibliography
% \newpage
% \bibliographystyle{acm}
% \bibliography{biblio}

\end{document}

